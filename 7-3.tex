\documentclass[letterpaper,12pt]{article}
%\documentclass[letterpaper,12pt,answers]{exam}
\usepackage{txfonts,multicol, fullpage, enumitem}
\usepackage{graphicx}
%\usepackage{../../../../std}
%\setlength{\voffset}{-0.5 in}
%\setlength{\hoffset}{-0.5 in}
%\setlength{\textwidth}{7.9 in}
%setlength{\oddsidemargin}{-0.8cm}
\setlength{\textheight}{10.5 in}
\setlength{\topmargin}{-1 in}
\setlength{\columnsep}{30pt}
\setlength{\headsep}{18pt}
\setlength{\parindent}{5pt}
\setlength{\parskip}{12pt}
%\setlength{\columnseprule}{1pt}
\DeclareGraphicsExtensions{.pdf, .eps, .jpg}

\raggedbottom
\renewcommand{\baselinestretch}{1}
\lefthyphenmin=4
\righthyphenmin=4

%\newcommand{\mb}{\mathbf}
%\newcommand{\qed}{\hfill$\Box$}
%\newcommand{\sol}{\item[]{\it Solution}:  }
%\newcommand{\ds}{\displaystyle}
%\renewcommand{\span}{\textrm{span}}
%\newcommand{\rref}{\textrm{rref}}
%\newcommand{\R}{\mathbb{R}}

\newcommand{\goes}{\rightarrow}
\newcommand{\mb}{\mathbf}
\newcommand{\qed}{\hfill$\Box$}
\newcommand{\sol}{\item[]{\it Solution}:  }
\newcommand{\ds}{\displaystyle}
\renewcommand{\span}{\textrm{span}}
\newcommand{\rref}{\textrm{rref}}
\newcommand{\R}{\mathbb{R}}
\newcommand{\rank}{\textrm{rank}}
\newcommand{\nullity}{\textrm{null}}
\renewcommand{\ker}{\textrm{ker}}
\newcommand{\im}{\textrm{im}}
\newcommand{\Z}{\mathbb{Z}}
\newcommand{\Q}{\mathbb{Q}}



\raggedbottom
\begin{document}
\centerline{}
\rightline{\large \bf Chapter 7.3 Worksheet \qquad\qquad\qquad Name \underline{\qquad\qquad\qquad\qquad\qquad}}\medskip
%\pagestyle{headandfoot}

\noindent \begin{tabular}{llc}
 & {\bf Objectives} & Rate your understanding of the objective \\ \hline

Objective 1 & Understand how to find a volume by &1\qquad 2\qquad 3 \qquad 4 \qquad 5 \\ & integrating the area of a cross section. &\\
Objective 2 & Calculate solids of revolution with the &1\qquad 2\qquad 3 \qquad 4 \qquad 5 \\ & shell method. &\\  \hline
\end{tabular}

\bigskip\bigskip

\centerline{\bf \large Warmup}
\noindent Write, but do not evaluate, an expression that finds the volume formed by revolving these regions around the y-axis. (By yourself)
\begin{enumerate}
\item The region under the curve $\ds y = \sin(x) + 1$ and before the line $x=2\pi$.
\item The region under the curve $\ds y = e^\frac{x}{2}$ and before the line $x=2$.
\item The region under the curve $\ds y = -x^2+ 4x + 5$ and before the line $x=5$.
\end{enumerate}

\centerline{\bf \large Problems}
\noindent These problems involve the shell method. Be sure to solve by integrating the area of a cylinder over the radius of the solid. You may reference Examples 209 through 212 in your example packet. (In groups)
\begin{enumerate}
\item Find the volume of a solid formed by revolving the curves $ y = \sin(x) + x $ and $ y = \sin(x) + 0.5\cdot x $ around the y-axis, bounded by $ x = 0 $ and $ x = 2\pi $.

\item {Consider the region between the curves $ y = x^2 $, $x=1$ and $\ds y=-x^2 $. Find the volume of the solid formed by revolving this region about the following lines.
\enumerate[label=(\alph*)]{
\begin{multicols}{7}
\item{$y=0$}
\item{$x=0$}
\item{$x=1$}
\item{$y=1$}
\item{$x=2$}
\item{$y=2$}
\item{$y=-1$}
\end{multicols}
}
}
\end{enumerate}

\bigskip
\centerline{\bf \large Self Quiz}
\centerline{(By yourself)}
\begin{enumerate}
\item Take the region under the curve $x=-y^3+3y^2-4y+4$ in the first quadrant. Find the volume of the solid formed by revolving this region about the x-axis.
\end{enumerate}

\bigskip
\centerline{\large \bf Reflection}
% this can be copied from the top of the worksheet
\noindent \begin{tabular}{llc}
{\bf Objectives} &  & Rate your understanding of the objective \\ \hline

Objective 1 & Understand how to find a volume by &1\qquad 2\qquad 3 \qquad 4 \qquad 5 \\ & integrating the area of a cross section. &\\
Objective 2 & Calculate solids of revolution with the &1\qquad 2\qquad 3 \qquad 4 \qquad 5 \\ & shell method. &\\  \hline\hline
\end{tabular}

\pagebreak

\noindent
Study Skills:
	\begin{itemize}
		\item Remember to read through examples from the book BEFORE your professor goes over the section in class.
		\item After class read through the examples in your notes from that day and try to do the problems yourself (without looking at your notes).
		\item After class read through the examples from the book in the section you JUST covered and make sure you understand them.
	\end{itemize}



\end{document}


