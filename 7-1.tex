\documentclass[letterpaper,12pt]{article}
%\documentclass[letterpaper,12pt,answers]{exam}
\usepackage{txfonts,multicol, fullpage}
\usepackage{graphicx}
%\usepackage{../../../../std}
%\setlength{\voffset}{-0.5 in}
%\setlength{\hoffset}{-0.5 in}
%\setlength{\textwidth}{7.9 in}
%setlength{\oddsidemargin}{-0.8cm}
\setlength{\textheight}{10.5 in}
\setlength{\topmargin}{-1 in}
\setlength{\columnsep}{30pt}
\setlength{\headsep}{18pt}
\setlength{\parindent}{5pt}
\setlength{\parskip}{12pt}
%\setlength{\columnseprule}{1pt}
\DeclareGraphicsExtensions{.pdf, .eps, .jpg}

\raggedbottom
\renewcommand{\baselinestretch}{1}
\lefthyphenmin=4
\righthyphenmin=4

%\newcommand{\mb}{\mathbf}
%\newcommand{\qed}{\hfill$\Box$}
%\newcommand{\sol}{\item[]{\it Solution}:  }
%\newcommand{\ds}{\displaystyle}
%\renewcommand{\span}{\textrm{span}}
%\newcommand{\rref}{\textrm{rref}}
%\newcommand{\R}{\mathbb{R}}

\newcommand{\goes}{\rightarrow}
\newcommand{\mb}{\mathbf}
\newcommand{\qed}{\hfill$\Box$}
\newcommand{\sol}{\item[]{\it Solution}:  }
\newcommand{\ds}{\displaystyle}
\renewcommand{\span}{\textrm{span}}
\newcommand{\rref}{\textrm{rref}}
\newcommand{\R}{\mathbb{R}}
\newcommand{\rank}{\textrm{rank}}
\newcommand{\nullity}{\textrm{null}}
\renewcommand{\ker}{\textrm{ker}}
\newcommand{\im}{\textrm{im}}
\newcommand{\Z}{\mathbb{Z}}
\newcommand{\Q}{\mathbb{Q}}



\raggedbottom
\begin{document}
\centerline{}
\rightline{\large \bf Chapter 7.1 Worksheet \qquad\qquad\qquad Name \underline{\qquad\qquad\qquad\qquad\qquad}}\medskip
%\pagestyle{headandfoot}

\noindent \begin{tabular}{llc}
 & {\bf Objectives} & Rate your understanding of the objective \\ \hline

Objective 1 & Express an area under a function on &1\qquad 2\qquad 3 \qquad 4 \qquad 5 \\ & the coordinate plane as an integral&\\
Objective 2 & Divide regions into ones that can be &1\qquad 2\qquad 3 \qquad 4 \qquad 5 \\ & expressed in terms of functions&\\
Objective 3 & Use both functions in terms of both &1\qquad 2\qquad 3 \qquad 4 \qquad 5 \\ & x and y & \\\hline
\end{tabular}

\bigskip\bigskip

\centerline{\bf \large Warmup}
\noindent Evaluate the following integrals. (By yourself)
\begin{enumerate}\begin{multicols}{3}
\item $\ds \int \frac{1}{x^2}dx$ 
\item $\ds \int _{0}^{5} 3x^2 + 6x + 3 dx$
\item $\ds \int _{3\pi}^{6\pi} \pi\cos(y)\sin(\pi\sin(y)) dy$
\end{multicols}
\end{enumerate}

\bigskip
\centerline{\bf \large Problems}
\noindent These problems involve areas on the coordinate plane. You may reference Example 199 in your example packet. (In groups)
\begin{enumerate}
\item Find the area that is both below $y = \sqrt{x}$ and above $y = x^2$ on the xy-plane. Graph the problem, and then write the lines as functions, and then use the intersection points as the limits of integration.
\item Find the area bounded by the line $ y = 2\pi $, the y axis, the curve $x = \sin (5y)$, and the x-axis.
\item Find the area of the region bounded by $y = -x^3 + 4x  $ and $ y = x^2 - 2x + 5 $ from $x = 0$ to $x = 2$. 
\item Find the area of the triangle formed by the points $(1,2), (5,1),$ and $(3,6)$. Solve again by rewriting your integrands and limits to integrate with respect to the other variable.
\item Find the area of the trapezoid formed by the points $(0,2)$, $(2,0)$, $(2,5)$, and $(0,3)$ . Check that the formula for the area produces the same result.
\item Solve the last problem using three integrals with respect to y. 
\end{enumerate}

\centerline{\bf \large Self Quiz}
\centerline{(By yourself)}
\begin{enumerate}
\item Calculate the area bounded by the functions $y=2x^2$ and $y=x^3$. 
\item Solve the last problem using an integral with respect to y.
\end{enumerate}

\centerline{\large \bf Reflection}
% this can be copied from the top of the worksheet
\noindent \begin{tabular}{llc}
{\bf Objectives} &  & Rate your understanding of the objective \\ \hline

Objective 1 & Express an area under a function on &1\qquad 2\qquad 3 \qquad 4 \qquad 5 \\ & the coordinate plane as an integral&\\
Objective 2 & Divide regions into ones that can be &1\qquad 2\qquad 3 \qquad 4 \qquad 5 \\ & expressed in terms of functions&\\
Objective 3 & Use both functions in terms of both &1\qquad 2\qquad 3 \qquad 4 \qquad 5 \\ & x and y & \\\hline\hline
\end{tabular}
\bigskip

\noindent
Study Skills:
	\begin{itemize}
		\item Remember to read through examples from the book BEFORE your professor goes over the section in class.
		\item After class read through the examples in your notes from that day and try to do the problems yourself (without looking at your notes).
		\item After class read through the examples from the book in the section you JUST covered and make sure you understand them.
	\end{itemize}



\end{document}


