\documentclass[letterpaper,12pt]{article}
%\documentclass[letterpaper,12pt,answers]{exam}
\usepackage{txfonts,multicol, fullpage, enumitem}
\usepackage{graphicx}
%\usepackage{../../../../std}
%\setlength{\voffset}{-0.5 in}
%\setlength{\hoffset}{-0.5 in}
%\setlength{\textwidth}{7.9 in}
%setlength{\oddsidemargin}{-0.8cm}
\setlength{\textheight}{10.5 in}
\setlength{\topmargin}{-1 in}
\setlength{\columnsep}{30pt}
\setlength{\headsep}{18pt}
\setlength{\parindent}{5pt}
\setlength{\parskip}{12pt}
%\setlength{\columnseprule}{1pt}
\DeclareGraphicsExtensions{.pdf, .eps, .jpg}

\raggedbottom
\renewcommand{\baselinestretch}{1}
\lefthyphenmin=4
\righthyphenmin=4

%\newcommand{\mb}{\mathbf}
%\newcommand{\qed}{\hfill$\Box$}
%\newcommand{\sol}{\item[]{\it Solution}:  }
%\newcommand{\ds}{\displaystyle}
%\renewcommand{\span}{\textrm{span}}
%\newcommand{\rref}{\textrm{rref}}
%\newcommand{\R}{\mathbb{R}}

\newcommand{\goes}{\rightarrow}
\newcommand{\mb}{\mathbf}
\newcommand{\qed}{\hfill$\Box$}
\newcommand{\sol}{\item[]{\it Solution}:  }
\newcommand{\ds}{\displaystyle}
\renewcommand{\span}{\textrm{span}}
\newcommand{\rref}{\textrm{rref}}
\newcommand{\R}{\mathbb{R}}
\newcommand{\rank}{\textrm{rank}}
\newcommand{\nullity}{\textrm{null}}
\renewcommand{\ker}{\textrm{ker}}
\newcommand{\im}{\textrm{im}}
\newcommand{\Z}{\mathbb{Z}}
\newcommand{\Q}{\mathbb{Q}}



\raggedbottom
\begin{document}
\centerline{}
\rightline{\large \bf Chapter 7.4 Worksheet \qquad\qquad\qquad Name \underline{\qquad\qquad\qquad\qquad\qquad}}\medskip
%\pagestyle{headandfoot}

\noindent \begin{tabular}{llc}
 & {\bf Objective} & Rate your understanding of the objective \\ \hline

Objective 1 & Understand how to set up and solve an &1\qquad 2\qquad 3 \qquad 4 \qquad 5 \\ & integral to finding the length of a curve. &\\ \hline
\end{tabular}

\bigskip\bigskip

\centerline{\bf \large Warmup}
\noindent Write, but do not evaluate, indefinite integrals that describe the length of these curves. It is not necessary to simplify beyond the square in the arc length formula (By yourself)
{
\enumerate{
\begin{multicols}{3}
\item {$y=x$}
\item {$y=\tan(x)$}
\item {$y=x^2+4x+4$}
\end{multicols}
}
}

\bigskip
\centerline{\bf \large Problems}
\noindent These problems involve arc length. You may reference Examples 213 and 214 in your example packet. (In groups)
\begin{enumerate}
\item Find the length of the curve $y=\sqrt{2x-2}$ contained within $\frac{3}{2}\leq x \leq 9$.
\item Find the length of the curve $y=\frac{x^2}{2}+1$ contained within $1\leq x \leq 4$. 
\item Write about why the last two answers were the same.
\item Find the length of the curve $y=(2x+3)^{\frac{3}{2}}$ contained in $0 \leq x \leq 5$.
\end{enumerate}

\bigskip
\bigskip
\centerline{\bf \large Self Quiz}
\centerline{(By yourself)}
\begin{enumerate}
\item A projectile is fired from the y-axis into the first quadrant. It follows the curve $y=-5x^2+10x+15$. How far does it travel before hitting the x-axis?
\end{enumerate}

\bigskip

\centerline{\large \bf Reflection}
% this can be copied from the top of the worksheet
\noindent \begin{tabular}{llc}
 & {\bf Objectives} & Rate your understanding of the objective \\ \hline

Objective 1 & Understand how to set up and solve an &1\qquad 2\qquad 3 \qquad 4 \qquad 5 \\ & integral to finding the length of a curve. &\\\hline
\end{tabular}
\bigskip

\noindent
Study Skills:
	\begin{itemize}
		\item Remember to read through examples from the book BEFORE your professor goes over the section in class.
		\item After class read through the examples in your notes from that day and try to do the problems yourself (without looking at your notes).
		\item After class read through the examples from the book in the section you JUST covered and make sure you understand them.
	\end{itemize}



\end{document}


