\documentclass[letterpaper,12pt]{article}
%\documentclass[letterpaper,12pt,answers]{exam}
\usepackage{txfonts,multicol, fullpage, enumitem}
\usepackage{graphicx}
%\usepackage{../../../../std}
%\setlength{\voffset}{-0.5 in}
%\setlength{\hoffset}{-0.5 in}
%\setlength{\textwidth}{7.9 in}
%setlength{\oddsidemargin}{-0.8cm}
\setlength{\textheight}{10.5 in}
\setlength{\topmargin}{-1 in}
\setlength{\columnsep}{30pt}
\setlength{\headsep}{18pt}
\setlength{\parindent}{5pt}
\setlength{\parskip}{12pt}
%\setlength{\columnseprule}{1pt}
\DeclareGraphicsExtensions{.pdf, .eps, .jpg}

\raggedbottom
\renewcommand{\baselinestretch}{1}
\lefthyphenmin=4
\righthyphenmin=4

%\newcommand{\mb}{\mathbf}
%\newcommand{\qed}{\hfill$\Box$}
%\newcommand{\sol}{\item[]{\it Solution}:  }
%\newcommand{\ds}{\displaystyle}
%\renewcommand{\span}{\textrm{span}}
%\newcommand{\rref}{\textrm{rref}}
%\newcommand{\R}{\mathbb{R}}

\newcommand{\goes}{\rightarrow}
\newcommand{\mb}{\mathbf}
\newcommand{\qed}{\hfill$\Box$}
\newcommand{\sol}{\item[]{\it Solution}:  }
\newcommand{\ds}{\displaystyle}
\renewcommand{\span}{\textrm{span}}
\newcommand{\rref}{\textrm{rref}}
\newcommand{\R}{\mathbb{R}}
\newcommand{\rank}{\textrm{rank}}
\newcommand{\nullity}{\textrm{null}}
\renewcommand{\ker}{\textrm{ker}}
\newcommand{\im}{\textrm{im}}
\newcommand{\Z}{\mathbb{Z}}
\newcommand{\Q}{\mathbb{Q}}



\raggedbottom
\begin{document}
\centerline{}
\rightline{\large \bf Chapter 7.2 Worksheet \qquad\qquad\qquad Name \underline{\qquad\qquad\qquad\qquad\qquad}}\medskip
%\pagestyle{headandfoot}

\noindent \begin{tabular}{llc}
 & {\bf Objectives} & Rate your understanding of the objective \\ \hline

Objective 1 & Understand how to find a volume by &1\qquad 2\qquad 3 \qquad 4 \qquad 5 \\ & integrating the area of a cross section. &\\
Objective 2 & Calculate solids of revolution with the &1\qquad 2\qquad 3 \qquad 4 \qquad 5 \\ & washer method. &\\
Objective 3 & Calculate solids of revolution with the &1\qquad 2\qquad 3 \qquad 4 \qquad 5 \\ & disk method. & \\ \hline
\end{tabular}

\bigskip\bigskip

\centerline{\bf \large Warmup}
\noindent Write, but do not evaluate, an integral that finds the volume formed by revolving these regions around these lines. (By yourself)
\begin{enumerate}
\item {The region under $y=-x^2+4$ in the first quadrant
\enumerate[label=(\alph*)]{
\begin{multicols}{3}
\item {The x-axis.}
\item {The y-axis.}
\item {$\ds x=2$}
\end{multicols}
}
}

\item {The region bounded by $y=x^3$, $y=0$, and $x=3$
\enumerate[label=(\alph*)]{
\begin{multicols}{3}
\item {The x-axis.}
\item {The y-axis.}
\item {$\ds x=5$}
\end{multicols}
}
}
\end{enumerate}

\bigskip
\centerline{\bf \large Problems}
\noindent These problems involve the disk method and the washer method. You may reference Examples 205 and 207 in your example packet. (In groups)
\begin{enumerate}
\item Find the volume of a solid formed by revolving the curves $ x = \sin(y) + y $ and $ x = \sin(y) + 0.5\cdot y $ around the y-axis, bounded by $ y = 0 $ and $ y = 2\pi $.

\item {Consider the region between the curves $ y = x^2 $, $x=1$ and $\ds y=0 $. Find the volume of the solid formed by revolving this region about the following lines.
\enumerate[label=(\alph*)]{
\begin{multicols}{7}
\item{$y=0$}
\item{$x=0$}
\item{$x=1$}
\item{$y=1$}
\item{$x=2$}
\item{$y=2$}
\item{$y=-1$}
\end{multicols}
}
}
\end{enumerate}

\bigskip
\bigskip
\centerline{\bf \large Self Quiz}
\centerline{(By yourself)}
\begin{enumerate}
\item Use the regions described in the Warm Up section and then revolve them around the line $y=4$. Find integrals for the volumes of the solids formed.
\end{enumerate}

\bigskip

\centerline{\large \bf Reflection}
% this can be copied from the top of the worksheet
\noindent \begin{tabular}{llc}
 & {\bf Objectives} & Rate your understanding of the objective \\ \hline

Objective 1 & Understand how to find a volume by &1\qquad 2\qquad 3 \qquad 4 \qquad 5 \\ & integrating the area of a cross section. &\\
Objective 2 & Calculate solids of revolution with the &1\qquad 2\qquad 3 \qquad 4 \qquad 5 \\ & washer method. &\\
Objective 3 & Calculate solids of revolution with the &1\qquad 2\qquad 3 \qquad 4 \qquad 5 \\ & disk method. & \\ \hline
\end{tabular}
\bigskip

\noindent
Study Skills:
	\begin{itemize}
		\item Remember to read through examples from the book BEFORE your professor goes over the section in class.
		\item After class read through the examples in your notes from that day and try to do the problems yourself (without looking at your notes).
		\item After class read through the examples from the book in the section you JUST covered and make sure you understand them.
	\end{itemize}



\end{document}


