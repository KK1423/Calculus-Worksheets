\documentclass[letterpaper,12pt]{article}
%\documentclass[letterpaper,12pt,answers]{exam}
\usepackage{txfonts,multicol, fullpage}
\usepackage{graphicx}
%\usepackage{../../../../std}
%\setlength{\voffset}{-0.5 in}
%\setlength{\hoffset}{-0.5 in}
%\setlength{\textwidth}{7.9 in}
%setlength{\oddsidemargin}{-0.8cm}
\setlength{\textheight}{10.5 in}
\setlength{\topmargin}{-1 in}
\setlength{\columnsep}{30pt}
\setlength{\headsep}{18pt}
\setlength{\parindent}{5pt}
\setlength{\parskip}{12pt}
%\setlength{\columnseprule}{1pt}
\DeclareGraphicsExtensions{.pdf, .eps, .jpg}

\raggedbottom
\renewcommand{\baselinestretch}{1}
\lefthyphenmin=4
\righthyphenmin=4

%\newcommand{\mb}{\mathbf}
%\newcommand{\qed}{\hfill$\Box$}
%\newcommand{\sol}{\item[]{\it Solution}:  }
%\newcommand{\ds}{\displaystyle}
%\renewcommand{\span}{\textrm{span}}
%\newcommand{\rref}{\textrm{rref}}
%\newcommand{\R}{\mathbb{R}}

\newcommand{\goes}{\rightarrow}
\newcommand{\mb}{\mathbf}
\newcommand{\qed}{\hfill$\Box$}
\newcommand{\sol}{\item[]{\it Solution}:  }
\newcommand{\ds}{\displaystyle}
\renewcommand{\span}{\textrm{span}}
\newcommand{\rref}{\textrm{rref}}
\newcommand{\R}{\mathbb{R}}
\newcommand{\rank}{\textrm{rank}}
\newcommand{\nullity}{\textrm{null}}
\renewcommand{\ker}{\textrm{ker}}
\newcommand{\im}{\textrm{im}}
\newcommand{\Z}{\mathbb{Z}}
\newcommand{\Q}{\mathbb{Q}}



\raggedbottom
\begin{document}
\centerline{}
\rightline{\large \bf Chapter 7.2 Worksheet \qquad\qquad\qquad Name \underline{\qquad\qquad\qquad\qquad\qquad}}\medskip
%\pagestyle{headandfoot}

\noindent \begin{tabular}{llc}
 & {\bf Objectives} & Rate your understanding of the objective \\ \hline

Objective 1 & Understand how to find a volume by &1\qquad 2\qquad 3 \qquad 4 \qquad 5 \\ & integrating the area of a cross section. &\\
Objective 2 & Calculate solids of revolution with the &1\qquad 2\qquad 3 \qquad 4 \qquad 5 \\ & washer method. &\\
Objective 3 & Calculate solids of revolution with the &1\qquad 2\qquad 3 \qquad 4 \qquad 5 \\ & disk method. & \\ \hline
\end{tabular}

\bigskip\bigskip

\centerline{\bf \large Warmup}
\noindent Evaluate the folowing integrals. (By yourself)
\begin{enumerate}\begin{multicols}{3}
\item $\ds \int e^{x}\sin(x) dx$
\item $\ds \int \frac{\ln(x)}{x^2} dx$
\item $\ds\sum_{n=1}^\infty (-1)^{n}\frac{n+1}{n^2+7n-1}$
\end{multicols}
\end{enumerate}

\centerline{\bf \large Problems}
\noindent These problems involve the definition of the derivative. You may reference Example 205 in your example packet. (By yourself)
\begin{enumerate}\begin{multicols}{3}
\item State the definition of the derivative.
\item $\ds\sum_{n=1}^\infty (-1)^{n}\frac{1}{n+7}$
\item $\ds\sum_{n=1}^\infty (-1)^{n}\frac{n+1}{n^2+7n-1}$
\end{multicols}
\end{enumerate}

\noindent These problems involve the definition of the derivative. You may reference Example XXX in your example packet. (Groups)
\begin{enumerate}
\item $\ds\sum_{n=1}^\infty (-1)^{n}\frac{n+1}{n^2+7n-1}$

\end{enumerate}

\noindent These problems involve the definition of the derivative. You may reference Example XXX in your example packet. (Groups)
\begin{enumerate}
\item $\ds\sum_{n=1}^\infty (-1)^{n}\frac{1}{n+7}$
\end{enumerate}


\noindent These problems involve the definition of the derivative. You may reference Example XXX in your example packet. (Self Quiz)
\begin{enumerate}\begin{multicols}{2}
\item $\ds\sum_{n=1}^\infty (-1)^{n+1}\frac{1}{n^2+7}$
\item $\ds\sum_{n=1}^\infty (-1)^{n}\frac{1}{n+7}$
\end{multicols}
\end{enumerate}

\vskip 2 in

\centerline{\large \bf Reflection}
% this can be copied from the top of the worksheet
\noindent \begin{tabular}{llc}
{\bf Objectives} &  & Rate your understanding of the objective \\ \hline

Objective 1 & write the objective here if it gets too long &1\qquad 2\qquad 3 \qquad 4 \qquad 5 \\ & continue it here &\\
Objective 2 & write the objective here &1\qquad 2\qquad 3 \qquad 4 \qquad 5 \\ & &\\
Objective 3 & write the objective here &1\qquad 2\qquad 3 \qquad 4 \qquad 5 \\  \hline\hline
\end{tabular}
\bigskip

\noindent
Study Skills:
	\begin{itemize}
		\item Remember to read through examples from the book BEFORE your professor goes over the section in class.
		\item After class read through the examples in your notes from that day and try to do the problems yourself (without looking at your notes).
		\item After class read through the examples from the book in the section you JUST covered and make sure you understand them.
	\end{itemize}



\end{document}


